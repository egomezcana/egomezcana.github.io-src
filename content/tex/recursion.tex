
\documentclass[letter,10pt,notitlepage]{amsart}
\usepackage[utf8]{inputenc}
\usepackage{amsthm}
\usepackage{amsmath}
\usepackage{amsfonts}
\usepackage{amssymb}
\usepackage{hyperref}
%\usepackage{amscd}
\usepackage[spanish]{babel}

%Tipografía distinta
%\usepackage{amsopn}
%\usepackage{mathrsfs}
%\usepackage{palatino}
%\usepackage{mathpazo}

\newtheorem{theorem}{Teorema}
\newtheorem{lemma}{Lema}
\newtheorem{proposition}{Proposición}
\newtheorem{corollary}{Corolario}

\theoremstyle{definition}
\newtheorem{definition}{Definición}
\newtheorem{axiom}{Axioma}
\newtheorem*{example}{Ejemplo}

\theoremstyle{remark}
\newtheorem*{remark}{Comentario}


\title{Recurrencia 2: Ecuaciones en diferencias y el teorema de recursión}
\date{9-may-2016}
\keywords{ecuaciones en diferencias, relaciones de recurrencia,recursión}

\begin{document}
\maketitle


Curioseando un poco más con las analogías entre los
problemas de ecuaciones diferenciales y ecuaciones en diferencias,
me encontré con una versión del teorema de recursión que
consigue solucionar el problema asociado a las ecuaciones
en diferencia.

\begin{theorem}[Teorema de recursión]
  Sea \( A\) un conjunto, \( g \colon A \times \mathbb{N} \to A\)
  una función y \( a\) un elemento de \( A\). Entonces,
  existe una única función \( f \colon \mathbb{N} \to A\) de
  forma que \[f(0) = a\] y \[ f(n+1)=g\left( f(n),n \right).\]
\end{theorem}
\begin{proof}
  Vamos a probar primero que 
  existe un subconjunto \( f \subset \mathbb{N} \times A \)
  el cual resulta una función. Para esto definimos el conjunto
  \[ \mathcal{C} = \left\{ X \subset \mathbb{N} \times A \mid (0,a) \in X \text{ y } (n,x)\in X\text{ implica }(n+1,g(x,n))\in X \right\}. \]
  Como \( \mathbb{N} \times A\) es un subconjunto de si mismo y contiene todas las parejas posibles, entonces
  \( \mathbb{N} \times A \in \mathcal{C}\) lo que nos permite
   concluir que el conjunto \( \mathcal{C} \) es no vacío. Entonces,
  definimos \[ f = \bigcap \mathcal{C} \] y afirmamos que satisface la propiedad que define a \( \mathcal{C} \). 
  En efecto, como \( (0,a)\) está en todos los elementos de \( \mathcal{C}\) entonces
  debe ser un miembro de la intersección y por tanto \( (0,a) \in f\). Ademas,
  si \( (n,x) \in f \), entonces para cada \( X \in \mathcal{C} \)
  la pareja \( (n,x) \in X \) y en ese caso \( (n+1,g(x,n))\in X \) de lo que podemos deducir 
  que \( (n+1,g(x,n)) \in f \). En otras palabras \( f \in \mathcal{C} \). También, por 
  la forma en que hemos definido a \( f\), si  \( v\in \mathcal{C} \) debemos tener  \( f \subseteq v \).
  Esta propiedad va a garantizar que \( f\) resulta ser una función lo
  cual mostraremos por inducción.
  
  Por definición \( (0,a) \in f\) garantizando que existe al menos un elemento
  relacionado con \( 0\) en \( f\). Basta mostrar que ese elemento
  es el único posible. Para esto, supongamos primero que \( (0,b)\) es un elemento
  de \( f\) de forma que \( a \neq b\). Entonces, el conjunto definido como 
  \( v = f \backslash\{(0,b)\}\) debe ser un subconjunto propio \( f\) que además 
  satisface las condiciones para ser un elemento de \( \mathcal{C}\). En efecto,
  por definición  \( (0,a) \in v\) y,
  si \( (n,x) \in v \), entonces \( (n,x) \in f \) por lo que  \( (n+1,g(x,n)) \in f \) pero
  \( (n+1,g(x,n))\neq (0,b) \) lo cual implica que \( (n+1,g(x)) \in v\).
  Sin embargo, esto es una contradicción pues, como se afirma en el
  párrafo anterior, \( f \subseteq v\). Entonces \( a = b\) como
  se buscaba.

  Supongamos ahora el resultado para \( n\), i.e., existe un
  elemento \( b\) de forma que \( (n,b) \in f\) siendo éste el único
  con dicha propiedad. Lo anterior implica que
  \( (n+1,g(b,n)) \in f\), por lo que existe un elemento relacionado
  con \( n+1\) en \( f\) y debemos mostrar que éste es el único
  con esta propiedad. Si suponemos que \( (n+1,c) \in f\)
  pero \( g(b,n) \neq c\), procedemos definiendo \( v = f \backslash\{(n+1,c)\}\) 
  el cual es un subconjunto propio de \( f\) y un elemento de \( \mathcal{C} \).
  En efecto, si \( (m,z) \in v \) entonces \( (m,z) \in f \) y tenemos dos posibilidades:
  \( m = n \) o \( m \neq n \). Para la primera, al tener \( (n,z) \in f \), la hipótesis
  de inducción garantiza que \( z = b \) y podemos afirmar sin más que 
  \( (m+1,g(z,m)) = (n+1,g(b,n)) \neq (n+1,c) \).
  Para la segunda, debemos tener que \( m+1 \neq n+1 \) y \( (m+1,g(z,m)) \neq (n+1,c) \).
  En cualquiera de los casos, es posible concluir \( (m+1,g(z,m)) \in v \) al ser 
  \( (m+1,g(z,m)) \neq (n+1,c)  \). Lo anterior afirma precisamente que \( v \in \mathcal{C} \). 
  Esto es por supuesto una contradicción pues \( f \subseteq v \),
  de acuerdo al primer párrafo de esta demostración. Debemos entonces
  concluir que \( g(b,n) = c\) como buscábamos.

  Por inducción, los resultados de los párrafos anteriores garantizan
  que para cada \( n\) en los naturales, existe un único elemento de \( A\)
  que lo acompaña en \( f\). Esto es lo mismo que decir que \( f\) es una función
  de acuerdo con la definición como relación. Resta probar que \( f \) cumple las propiedades que pide 
  el teorema. De entrada, por la definición  de \( \mathcal{C} \), tenemos \( (0,a) \in f\) lo
  cual se expresa como \( f(0) = a \). Ahora, \( f(n) \) debe cumplir que
  \( (n, f(n)) \in f \) y como \( f \in \mathcal{C} \), entonces \( (n+1, g(f(n),n)) \in f \) o lo
  que es lo mismo \( f(n+1) = g(f(n),n) \). Con esto hemos mostrado que existe una
  función con las propiedades citadas.

  Supongamos ahora que \( f_1\) es una función que cumple con las mismas propiedades.
  Mostraremos que \( f_1 = f\) usando de nueva cuenta inducción. Para el caso
  base es inmediato que
  \[ f_1(0) = a = f(0).\]
  Supongamos ahora \( f_1(n) = f(n)\), en ese caso
  \[ f_1(n+1) = g(f_1(n),n) = g(f(n),n) = f(n+1).\]
  Lo anterior muestra que \( f_1 = f\) mostrando que \( f\) es la única función con las propiedades
  que buscamos. Q.E.D.
\end{proof}

El teorema de recursión, así plateado, no parece resolver el problema que hemos
definido \href{{filename}/posts/recurrencia.md}{aquí} como ecuaciones en diferencias. 
Sin embargo, no es difícil modificarlo para obtener lo que buscamos.

\begin{theorem}
  Sea \( A\) un conjunto, \( g \colon A^{k} \times \mathbb{N} \to A\) una función,
  y sean también \( a_0,a_1,\dots,a_{k-2}\) y \( a_{k-1}\)  elementos cualquiera de \( A\). Entonces,
  existe una única función \( f \colon \mathbb{N} \to A\) de forma
  que \( f(0) = a_0\), \( f(1) = a_1\), \dots, \( f(k-1) = a_{k-1}\)
  y para todo número natural \( n\)
  \[ f(n+k) = g\left( f(n),\dots,f(n+k-1),n \right)\]
\end{theorem}
\begin{proof}
  Sea \( b_0 = (a_0,\dots,a_{k-1})\) y sea también 
  \( G \colon A^{k} \times \mathbb{N} \to A^{k}\) la función definida  como
  \[ G(x_0,\dots,x_{k-1},n) = (x_1,\dots,x_{k-1}, g\left( x_0,\dots,x_{k-1},n\right)).\]
  Según el teorema de recursión, existe una única función \( h \colon \mathbb{N} \to A^{k}\)
  de forma que
  \[ h(0) = b_0\]
  y
  \[ h(n+1) = G(h(n),n).\]
  Usando esta función, definimos entonces \( f \colon \mathbb{N} \to A\) de la siguiente manera:
  \[ f(n) = 
    \begin{cases}
      a_n & n < k \\
      g(h(n-k),n-k) & n \geq k
    \end{cases}.
  \]
  Afirmamos que esta función satisface, para todo natural \( n\),
  \[ h(n) = (f(n), \dots, f(n+k-1)).  \]
  En efecto, podemos observar que
  \[ h(0) = (a_0,\dots,a_{k-1}) = (f(0),\dots,f(k-1)).\]
  y si suponemos que \(h(n) = (f(n), \dots, f(n+k-1))\),
  entonces
  \begin{align*}
    h(n+1) &= G(h(n),n) \\
       &= \left( f(n+1),\dots,f(n+k-1), g(h(n),n) \right) \\
       &= \left( f(n+1),\dots,f(n+k-1), f(n+k) \right).
  \end{align*}
  Usando inducción, lo anterior muestra lo que 
  afirmamos. Ahora, con esa información, se desprende inmediatamente que
  \begin{align*}
    f(n+k) &= g(h(n),n) \\
    &= g(f(n),\dots,f(n+k-1),n),
  \end{align*}
  lo que garantiza la existencia de la función con las propiedades
  que afirma el teorema.

  Por último, supongamos que \( f_1\) es una función con las
  mismas propiedades. Usando inducción probaremos que son iguales
  y para eso basta mostrar para todo natural \( n\) que
  \[ f_1(n+k) = f(n+k).\] Para el caso base, debemos observar que
  \[ f_1(k) = g(f(0),\dots,f(n+k-1),0) = f(k)\]
  y si \( f_1(m+k) = f(m+k)\) para todo \( m \leq n\),
  entonces
  \begin{align*}
    f_1(n+1+k) &= g(f_1(n+1),\dots,f_1(n+k),n+1) \\
     &= g(f(n+1),\dots,f(n+k),n+1) \\
     &= f(n+1+k).
  \end{align*}
  Por inducción fuerte, tenemos \( f_1 = f\), mostrando
  que \( f\) es la única función con las propiedades citadas. Q.E.D.
\end{proof}

El anterior teorema se puede ver como una versión
del teorema de recursión (de hecho, los enunciados son equivalentes)
con la peculiaridad de dar una respuesta positiva al problema de las 
ecuaciones en diferencia. Bajo la mirada adecuada, es la versión
discreta del teorema de Picard-Lindelöf y a diferencia de éste, las 
soluciones se pueden garantizar de manera general a base de dar 
condiciones iniciales. Esto implicaría que cualquier método que
consiguiera encontrar una solución para una ecuación en diferencias,
encontraría la única solución posible indicada en el teorema.

\end{document}


