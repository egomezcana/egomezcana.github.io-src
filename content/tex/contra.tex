\documentclass[draft,letter,10pt,notitlepage]{amsart}
\usepackage[utf8]{inputenc}
\usepackage{amsthm}
\usepackage{amsmath}
\usepackage{amsfonts}
\usepackage{amssymb}
\usepackage{amscd}
\usepackage[spanish]{babel}

\newtheorem{theorem}{Teorema}
\newtheorem{lemma}{Lema}
\newtheorem{proposition}{Proposición}
\newtheorem{corollary}{Corolario}

\theoremstyle{definition}
\newtheorem*{definition}{Definición}
\newtheorem{axiom}{Axioma}
\newtheorem*{example}{Ejemplo}

\theoremstyle{remark}
\newtheorem*{remark}{Comentario}


\title{Patología en una categoría de espacios métricos.}
\date{}
\keywords{Categorías,espacios métricos}

\begin{document}
\maketitle
\begin{abstract}
  Esto puede funcionar bien
\end{abstract}

Si algo me ha perturbado en aprender teoría de categorías, es 
la abrumadora cantidad de definiciones que hay y de las que
en un umbral de absoluta ignorancia, pocos contraejemplos
conozco. Una de ellas, es el concepto de cateogoría balanceada.

\begin{definition}
  Una categoría \( \mathcal{C}\) se dice \emph{balanceada}
  si cualquier morfismo \( f \colon A \to B\) en la categoría
  que sea al mismo tiempo un monomorfismo y epimorfismo
  es un isomorfismo.
\end{definition}

Muchas categorías bien conocidas son balanceadas: \( \mathbf{Set}\),
\( \mathbf{Pos}\), \( \mathbf{Grp}\), etc. Se necesita sentarse
a pensar si habrá categorías no balanceadas y un ejemplo bastante
lindo resulta la categoría \( \mathbf{Met}\) formada por
los espacios métricos como sus objetos y las funciones
1-Lipschitz como sus morfismos.

\end{document}


