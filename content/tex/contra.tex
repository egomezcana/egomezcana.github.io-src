\documentclass[letter,10pt,notitlepage]{amsart}
\usepackage[utf8]{inputenc}
\usepackage{amsthm}
\usepackage{amsmath}
\usepackage{amsfonts}
\usepackage{amssymb}
%\usepackage{amscd}
\usepackage[spanish]{babel}

\newtheorem{theorem}{Teorema}
\newtheorem{lemma}{Lema}
\newtheorem{proposition}{Proposición}
\newtheorem{corollary}{Corolario}

\theoremstyle{definition}
\newtheorem*{definition}{Definición}
\newtheorem{axiom}{Axioma}
\newtheorem*{example}{Ejemplo}

\theoremstyle{remark}
\newtheorem*{remark}{Comentario}



\title{Contraejemplos: Categorías balanceadas}
\date{5-5-2016}
\keywords{Categorías,espacios métricos,contraejemplos}

\begin{document}
\maketitle

Si algo me ha perturbado en aprender teoría de categorías, es 
la abrumadora cantidad de definiciones que hay y de las que
en un umbral de absoluta ignorancia, pocos contraejemplos
conozco. Una de ellas, es el concepto de categoría balanceada.

\begin{definition}
  Una categoría \( \mathcal{C}\) se dice \emph{balanceada}
  si cualquier morfismo \( f \colon A \to B\) en la categoría
  que sea al mismo tiempo un monomorfismo y epimorfismo
  es un isomorfismo.
\end{definition}

Muchas categorías bien conocidas son balanceadas: \( \mathbf{Set}\),
\( \mathbf{Pos}\), \( \mathbf{Grp}\), etc. Se necesita sentarse
a pensar si habrá categorías no balanceadas y un ejemplo bastante
lindo resulta la categoría \( \mathbf{Met}\) formada por
los espacios métricos como sus objetos y los morfismos métricos
como sus morfismos.

\begin{definition}
  Sean \( A\) y \( B\) espacios métricos y sea
  \( f \colon A \to B\). Entonces, \( f\) se dice \emph{un
  morfismo métrico} si 
  \[ d_B(f(x),f(y)) \leq d_A(x,y).\]
\end{definition}

Para determinar si es o no balanceada necesitamos caracterizar
los isomorfismos, lo mismo que los monomorfismos y epimorfismos en 
\( \mathbf{Met}\). 

\begin{lemma}
  Sean \( A\) y \( B\) espacios métricos y sea \( f \colon A \to B\)
  un morfismo métrico. Entonces, \( f\) es un isomorfismo
  si y sólo si  \( f\) es biyectiva y una isometría.
\end{lemma}
\begin{proof}
  Por un lado, si \( f\) es un isomorfimo, entonces es invertible y 
  su inversa es un morfismo métrico. Esto implica que
  \begin{align*}
    d_A(x,y) &= d_A\left( f^{-1}(f(x)), f^{-1}(f(y)) \right) \\
    	&\leq d_B(f(x),f(y)) \\
	& \leq d_A(x,y)
  \end{align*}
  por tanto \( d_A(x,y) = d_B(f(x),f(y))\). Lo anterior muestra que \( f\)
  es una isometría como buscábamos. Ahora, si \( f\) es biyectiva y 
  una isometría, entonces su inversa debe ser de igual forma una
  isometría y por tanto un morfismo métrico; en otras palabras
  es un isomorfismo en \( \mathbf{Met}\). Q.E.D.
\end{proof}

\begin{lemma}
  Sean \( A\) y \( B\) espacios métricos y sea \( f \colon A \to B\)
  un morfismo métrico. Entonces, \( f\) es un monomorfismo si
  \( f\) es inyectiva.
\end{lemma}
\begin{proof}
  Al ser \( f\) inyectiva, \( f \circ g_1 = f \circ g_2\) implica
  que \( g_1 = g_2\), resultando un monomorfismo. Q.E.D.
\end{proof}

\begin{lemma}
  Sean \( A\) y \( B\) espacios métricos y sea \( f \colon A \to B\)
  un morfismo métrico. Entonces \( f\) es un epimorfismo si
  el conjunto \( \mathrm{im}(f)\) es denso.
\end{lemma}
\begin{proof}
  Supongamos que \( g_1,g_2 \colon B \to C\) son morfismos métricos de forma
  que \( g_1 \circ f = g_2 \circ f\). Debemos probar que \( g_1 = g_2\) y
  para conseguir esto, tomamos \( x \in B\) y \( \varepsilon>0\) y afirmamos
  que \( d_C(g_1(x),g_2(x)) < \varepsilon\). En efecto, como
  la imagen es densa, entonces existe \( y \in A\) de forma que
  \[ d_B(f(y),x) < \frac{\varepsilon}{2},\]
  de esta forma podemos tomar \( z = g_1(f(a)) = g_2(f(a))\) y en
  consecuencia
  \begin{align*}
    d_C(g_1(x),g_2(x)) &\leq d_C(g_1(x), z) + d_C(z,g_2(x)) \\
    &\leq d_B(x,f(y)) + d_B(f(y),x) \\
    &< \varepsilon,
  \end{align*}
  mostrando con esto nuestra afirmación. Ahora, 
  como la elección de \( \varepsilon\) fue arbitraria, podemos
  concluir que \( d_C(g_1(x),g_2(x)) = 0\) o en otras palabras
  \( g_1(x) = g_2(x)\). Siendo la elección de \( x\) arbitraria,
  podemos también concluir que \( g_1 \circ f = g_2 \circ f\)
  implica \( g_1 = g_2\) mostrando que \( f\) es un epimorfismo. Q.E.D.
\end{proof}

Basta ahora considerar la inclusión \( i \colon \mathbb{Q} \to \mathbb{R}\)
la cual es un morfismo métrico. Como es inyectiva, lo anterior nos
permite concluir que debe ser un monomorfismo, además la imagen de \( i\)
es densa y en consecuencia un epimorfismo.  Sin embargo, 
\( i\) no es biyectiva por lo que no puede ser un isomorfismo. Esto
quiere decir que existen morfismos métricos que resultan monomorfismos
y epimorfismos pero no isomorfismos. En otras palabras:

\begin{theorem}
  La categoría \( \mathbf{Met}\) no es balanceada.
\end{theorem}

\end{document}

