\documentclass[draft,letter,10pt,notitlepage]{amsart}
\usepackage[utf8]{inputenc}
\usepackage{amsthm}
\usepackage{amsmath}
\usepackage{amsfonts}
\usepackage{amssymb}
\usepackage{amscd}
\usepackage[spanish]{babel}

\newtheorem{theorem}{Teorema}
\newtheorem{lemma}{Lema}
\newtheorem{proposition}{Proposición}
\newtheorem{corollary}{Corolario}

\theoremstyle{definition}
\newtheorem*{definition}{Definición}
\newtheorem{axiom}{Axioma}
\newtheorem*{example}{Ejemplo}

\theoremstyle{remark}
\newtheorem*{remark}{Comentario}


\title{Espacios métricos como contraejemplos I}
\date{}
\keywords{Categorías,espacios métricos}

\begin{document}
\maketitle

Si algo me ha perturbado en aprender teoría de categorías, es 
la abrumadora cantidad de definiciones que hay y de las que
en un umbral de absoluta ignorancia, pocos contraejemplos
conozco. Una de ellas, es el concepto de cateogoría balanceada.

\begin{definition}
  Una categoría \( \mathcal{C}\) se dice \emph{balanceada}
  si cualquier morfismo \( f \colon A \to B\) en la categoría
  que sea al mismo tiempo un monomorfismo y epimorfismo
  es un isomorfismo.
\end{definition}

Muchas categorías bien conocidas son balanceadas: \( \mathbf{Set}\),
\( \mathbf{Pos}\), \( \mathbf{Grp}\), etc. Se necesita sentarse
a pensar si habrá categorías no balanceadas y un ejemplo bastante
lindo resulta la categoría \( \mathbf{Met}\) formada por
los espacios métricos como sus objetos y los morfismos métricos
como sus morfismos.

\begin{definition}
  Sean \( A\) y \( B\) espacios métricos y sea
  \( f \colon A \to B\). Entonces, \( f\) se dice \emph{un
  morfismo métrico} si 
  \[ d_B(f(x),f(y)) \leq d_A(x,y).\]
\end{definition}

Para determinar si es o no balanceada necesitamos caracterizar
los isomorfismos, lo mismo que los monomorfismos y epimorfismos en 
\( \mathbf{Met}\). 

\begin{lemma}
  Sean \( A\) y \( B\) espacios métricos y sea \( f \colon A \to B\)
  un morfismo métrico. Entonces, \( f\) es un isomorfismo
  si y sólo si  \( f\) es biyectiva y una isometría.
\end{lemma}
\begin{proof}
  Por un lado, si \( f\) es un isomorfimo, entonces es invertible y 
  su inversa es un morfismo métrico. Esto implica que
  \begin{align*}
    d_A(x,y) &= d_A\left( f^{-1}(f(x)), f^{-1}(f(y)) \right) \\
    	&\leq d_B(f(x),f(y)) \\
	& \leq d_A(x,y)
  \end{align*}
  por lo que \( d_A(x,y) = d_B(f(x),f(y))\), mostrando que \( f\)
  es una isometría como buscábamos. Ahora, si \( f\) es biyectiva y 
  una isometría, entonces su inversa debe ser de igual forma una
  isometría y por tanto un morfismo métrico; en otras palabras
  es un isomorfismo en \( \mathbf{Met}\).
\end{proof}

\begin{lemma}
  Sean \( A\) y \( B\) espacios métricos y sea \( f \colon A \to B\)
  un morfismo métrico. Entonces, \( f\) es un monomorfismo si
  \( f\) es inyectiva.
\end{lemma}
\begin{proof}
  Al ser \( f\) inyectiva, \( f \circ g_1 = f \circ g_2\) implica
  que \( g_1 = g_2\), resultando un monomorfismo.
\end{proof}

\begin{lemma}
  Sean \( A\) y \( B\) espacios métricos y sea \( f \colon A \to B\)
  un morfismo métrico. Entonces \( f\) es un epimorfismo si
  el conjunto \( \mathrm{im}(f)\) es denso.
\end{lemma}
\begin{proof}
  Supongamos que \( g_1,g_2 \colon B \to C\) son morfismos métricos de forma
  que \( g_1 \circ f = g_2 \circ f\). Debemos probar que \( g_1 = g_2\) y
  para conseguir esto tomamos \( b \in B\) y \( \varepsilon>0\). Como
  la imagen es densa, entonces existe \( a \in A\) de forma que
  \[ d_B(f(a),b) < \frac{\varepsilon}{2},\]
  de esta forma podemos tomar \( s = g_1(f(a)) = g_2(f(a))\) y en
  consecuencia
  \begin{align*}
    d_C(g_1(b),g_2(b)) &\leq d_C(g_1(b), s) + d_C(s,g_2(b)) \\
    &\leq d_B(b,f(a)) + d_B(f(a),b) \\
    &< \varepsilon.
  \end{align*}
  Como la elección de \( \varepsilon\) fue arbitraria, podemos
  concluir que \( d_C(g_1(b),g_2(b)) = 0\) o en otras palabras
  \( g_1(b) = g_2(b)\). En conclusión, \( g_1 \circ f = g_2 \circ f\)
  implica \( g_1 = g_2\) mostrando que \( f\) es un epimorfismo.
\end{proof}

Basta ahora considerar la inclusión \( i \colon \mathbb{Q} \to \mathbb{R}\)
la cual es un morfismo métrico. Como es inyectiva, lo anterior nos
permite concluir que debe ser un monomorfismo, además la imagen de \( i\)
es densa por lo que debe ser de igual forma un epimorfismo.  Sin embargo,
sin embargo, \( i\) no es una transformación congruente al no ser biyectiva.
Lo anterior permite concluir que:


\begin{theorem}
  La categoría \( \mathbf{Met}\) no es balanceada.
\end{theorem}

\end{document}

