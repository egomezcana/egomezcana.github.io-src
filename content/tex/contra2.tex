\documentclass[letter,10pt,notitlepage]{amsart}
\usepackage[utf8]{inputenc}
\usepackage{amsthm}
\usepackage{amsmath}
\usepackage{amsfonts}
\usepackage{amssymb}
\usepackage{amscd}
\usepackage[spanish]{babel}

\newtheorem{theorem}{Teorema}
\newtheorem{lemma}{Lema}
\newtheorem{proposition}{Proposición}
\newtheorem{corollary}{Corolario}

\theoremstyle{definition}
\newtheorem*{definition}{Definición}
\newtheorem{axiom}{Axioma}
\newtheorem*{example}{Ejemplo}

\theoremstyle{remark}
\newtheorem*{remark}{Comentario}


\title{Contraejemplos: Categorías con productos infinitos}
\date{10-9-2016}
\keywords{Categorías,espacios métricos,contraejemplos}

\begin{document}
\maketitle

Muchas de las categorías más comunes tienen productos finitos. La
pregunta que muchas veces nos acosa es: ¿Existen categorías con todos
los productos finitos pero no que tengan algunos productos infinitos?
La respuesta es sí, existen esas categorías. Usaremos la categoría de
los espeacio métricos, \( \mathbf{Met}\), para mostrar que esto es
posible. El resultado que exhibiremos es en realidad peor: La
existencia de productos finitos no conduce a la existencia productos
contables.

Comencemos notando que \(\mathbf{Met}\) tiene a todos los productos
finitos y en particular tiene un objeto terminal \(1=\{\ast\}\). Este
objeto es simplemente el espacio métrico con un elemento el cual para
cualquier espacio métrico \(X\) y cualquier elemento \(x_0 \in X\),
admite un morfismo métrico \( \kappa_{x_0} \colon \ast \mapsto x_0 \).

\begin{definition}
  Tomaremos \(2_{k}\) como el espacio métrico formado por \( \{0,1\}\)
  como su conjunto base y con métrica
  \[d_k(0,1) = k.\]
\end{definition}

Vamos a considerar la secuencia de espacios métricos
\[ \{2_k\}_{k \in \mathbb{N}}.\] Si la categoría tuviera todos los
productos infinitos entonces tendría el producto de la secuencia
anterior. Supongamos que este es el caso y que el producto es el
objeto \(P\) junto a los morfimos \(\pi_k \colon P \to 2_k\). Además,
para cada \(k\) podemos tomar los morfismos
\(\kappa_{0,k} \colon \ast \mapsto 0\) de forma que \(1\) y estos morfismos
forman un diagrama de producto y lo mismo sucede con los morfismos
\(\kappa_{1,k} \colon \ast \mapsto 1\). Esto implica que existen
morfismos \(f,g \colon 1 \to P\) únicos de forma que el siguiente
diagrama conmuta
\begin{equation*}
  \begin{CD}
    1 @>\kappa_{0,k}>>    2_k \\
    @VfVV     @| \\
    P @> \pi_k >>    2_k \\
    @AgAA    @| \\
    1 @>\kappa_{1,k}>>    2_k
  \end{CD}
\end{equation*}
Con las funciones \(f\) y \(g\) definiremos los elementos de \(a,b \in
P\) tomando estos como \(a = f(\ast)\) y \(b = g(\ast)\), para los que
siempre podremos encontrar un natural \(n\) que satisface
\[ n > d_P(a,b).\]
Lo cual es contradictorio pues tendríamos para todo \(k\)
\[
  k = d_k(0,1)
  =  d_k((\pi_k \circ f)(\ast), (\pi_k \circ f)(\ast))
  \leq d_P(a,b) < n.
\]
Debemos entonces concluir que no es posible encontrar un objeto \(P\)
que sea el producto de la secuencia de espacios métricos que
indicamos. En conclusión:

\begin{theorem}
  La categoría \( \mathbf{Met}\) no tiene todos los productos
  infinitos.
\end{theorem}

\end{document}


%%% Local Variables:
%%% mode: latex
%%% TeX-master: t
%%% End:
