
\documentclass[letter,10pt,notitlepage]{amsart}
\usepackage[utf8]{inputenc}
\usepackage{amsthm}
\usepackage{amsmath}
\usepackage{amsfonts}
\usepackage{amssymb}
%\usepackage{amscd}
\usepackage[spanish]{babel}

%Tipografía distinta
%\usepackage{amsopn}
%\usepackage{mathrsfs}
%\usepackage{palatino}
%\usepackage{mathpazo}

\newtheorem{theorem}{Teorema}
\newtheorem{lemma}{Lema}
\newtheorem{proposition}{Proposición}
\newtheorem{corollary}{Corolario}

\theoremstyle{definition}
\newtheorem{definition}{Definición}
\newtheorem{axiom}{Axioma}
\newtheorem*{example}{Ejemplo}

\theoremstyle{remark}
\newtheorem*{remark}{Comentario}


\title{El problema de las relaciones de recurrencia}
\date{3-May-2016}
\keywords{ecuaciones en diferencias,relaciones de recurrencia}


\begin{document}
\begin{abstract}
  Una breve exploración de la formulación de
  una ecuación en diferencias como problema
  y en analogía a una ecuación diferencial.
\end{abstract}
\maketitle

Curioseando un poco en temas de matemáticas
discretas, encontré poca información acerca de
como definir correctamente una relación de recurrencia
también llamada ecuación en diferencias. Me pareció útil 
establecer una analogía entre las ecuaciones diferenciales
y las ecuaciones en diferencias, usando como punto de partida 
las primeras como  problema.

\begin{definition}
  Sea \( g \colon \mathbb{R}^{k+2} \to \mathbb{R}\). \emph{Una ecuación diferencial ordinaria}
  es el problema de encontrar una función \( f \colon I \subseteq \mathbb{R} \to \mathbb{R}\)
  en \( I\), de forma que para todo \( x\) en el dominio de \( f\)
  \[ g\left( f^{(0)}(x),\dots,f^{(k)}(x),x \right) = 0.\]
  La ecuación anterior se dice estar \emph{en forma implícita};
  en contraste, la ecuación se dice estar \emph{en forma explícita}, si para una función
  \( g \colon \mathbb{R}^k \to \mathbb{R}\), podemos expresar el problema
  como\[ f^{(k)}(x) = g\left( f^{(0)}(x),\dots,f^{(k-1)}(x),x \right).\]
\end{definition}

Usando esta definición como base e inspiración, podemos formular un problema
discreto análogo. Habrá que admitir, de momento, que los reales juegan un rol
importante en la formulación.

\begin{definition}
  Sea \( g \colon \mathbb{R}^{k+1} \times \mathbb{N} \to \mathbb{R}\).
  \emph{Una ecuación en diferencias ordinaria} es el problema de encontrar una
  sucesión \( f \colon \mathbb{N} \to \mathbb{R}\) de forma que
  para todo \( n \in \mathbb{N}\),
  \[ g(f(n), f(n+1), \dots, f(n+k-1),n) = 0.\]
  La ecuación anterior se dice estar \emph{en forma implícita}; en contraste
  se dice estar \emph{en forma explícita}, si para una función 
  \( g \colon \mathbb{R}^k \times \mathbb{N} \to \mathbb{R}\), 
  podemos expresar el problema como
  \[ f(n+k)=g( f(n), \dots, f(n+k-1),n).\]
\end{definition}

El problema definido así, es la versión discreta de una ecuación
diferencial ordinaria. Una inspección más detenida nos deja ver
que los reales juegan un rol importante en las ecuaciones diferenciales
por toda la estructura que acarrean, sin embargo en el caso discreto
parecen ser inconsecuentes a menos que se quiera resolver el caso
continuo a través de una aproximación usando el caso discreto.
Parece entonces razonable librarnos de \( \mathbb{R}\) y
establecer un caso más general.

\begin{definition}
  Sea \( A\) conjunto cualquiera y sea \( g \colon A^{k+1} \times \mathbb{N} \to A\).
  \emph{Una ecuación en diferencias ordinaria sobre \( A\)} es el 
  problema de encontrar una sucesión \( f \colon \mathbb{N} \to A\) 
  de forma que para todo \( n \in \mathbb{N}\),
  \[ g(f(n), \dots, f(n+k),n) = 0.\]
\end{definition}

La igualdad \( g(f(n), \dots, f(n+k),n) = 0\)
establece una relación entre las entradas de la sucesión \( f\) y ésta
parece la razón del término \emph{relación de recurrencia}. Planteado
así, y aunque las expresiones \emph{ecuación en diferencias} y 
\emph{relación de recurrrencia} son prácticamente intercambiables, resultan
ligeramente distintas.

\begin{example}
  La ecuación \[ f(n+1) = (n+1)f(n)\] con condición inicial \( f(0)=1\)
  tiene como solución al factorial.
\end{example}

\begin{example}
  La ecuación \[ f(n+2) = f(n+1) + f(n)\] con condiciones iniciales
  \( f(0) = 1\) y \( f(1) = 1\) tiene como solución a la sucesión de Fibonacci.
\end{example}

Continuando con la analogía, podemos ir todavía más lejos definiendo 
``una ecuación en diferencias parciales'' tomando sucesiones dobles como
las potenciales soluciones.

\begin{definition}
  Sea \( g \colon A^{(k+1)(l+1)} \times \mathbb{N}^2 \to \mathbb{A}\).
  \emph{Una ecuación en diferencias parciales en dos variables sobre \( A\)} es el 
  problema de encontrar una doble sucesión \( u \colon \mathbb{N}^2 \to A\) 
  de forma que para todo \( n,m \in \mathbb{N}\),
  \[ g\left(u(n+k)(m+l),\dots,u(n,m),n,m\right) = 0.\]
\end{definition}


\begin{example}
  La ecuación en diferencias parciales \[ u(n+1)(m+1) = -nu(n,m+1) + u(n,m)\]
  con condiciones iniciales \( u(0,0) = 1\) y \( u(0,n) = u(n,0) = 0\)
  tiene como solución a los números de Stirling del primer tipo.
\end{example}

Sería interesante poder derivar alguna clase de teoremas de existencia
y unicidad para las ecuaciones en diferencias parecidos al de
Picard-Lindelöf.
\end{document}


