
\documentclass[letter,10pt,notitlepage]{amsart}
\usepackage[utf8]{inputenc}
\usepackage{amsthm}
\usepackage{amsmath}
\usepackage{amsfonts}
\usepackage{amssymb}
\usepackage[spanish]{babel}
%\usepackage{geometry}

\newtheorem{theorem}{Teorema}
\newtheorem{lemma}{Lema}
\newtheorem{proposition}{Proposición}
\newtheorem{corollary}{Corolario}

\theoremstyle{definition}
\newtheorem{definition}{Definición}
\newtheorem{axiom}{Axioma}
\newtheorem*{example}{Ejemplo}

\theoremstyle{remark}
\newtheorem*{remark}{Comentario}


\title{Series formales de potencia: Definiciones y un ejemplo }
\date{10-mar-2016}
\keywords{Series formales,anillos}

\begin{document}
\begin{abstract}
  Es notable que en el marco de las series
  formales de potencia, puedan aparecer
  objetos formales que de alguna forma representan
  las aproximaciones de funciones. Con este tipo
  de resultados podemos manipular las series
  de potencia de manera algebraica olvidándonos 
  por un momento de la indumentaria analítica.
\end{abstract}
\maketitle

Hay una idea muy interesante cuando se
estudia el anillo de series formales de potencia:
La existencia de unidades. Estas unidades nos 
permiten dar una versión algebraica de la aproximación
de algunas funciones sin necesidad de hablar de
convergencia.

Consideraremos para esto \( R\) 
como un anillo conmutativo y por cuestiones puramente 
sentimentales, asumiremos que los anillos siempre vienen 
acompañados de un elemento unitario (sentimentalismo categórico). 
Se construirá sobre este anillo el denominado anillo 
de las series formales de potencia.

\begin{definition}
  Una función \( \mathbb{N} \to R \) se dirá una 
  \emph{sucesión sobre \( R\)}. Al conjunto de todas 
  las sucesiones sobre \(R\), se le denotará como 
  \( R[[x]]\).
\end{definition}

Para una sucesión \(a\) sobre \(R\), se acostumbra 
escribir \(a_k = a(k)\). Es también contrumbre representar
a la sucesión como una tupla con infinito números de 
entradas:
\[ a = (a_0,a_1, \dots ).\]
Sin embargo, existe una caracterización en la que tenemos 
más interés de momento. Necesitamos definir un par de 
operaciones antes de continuar.

\begin{definition}
  \label{OP}
  Para sucesiones \(a\) y \(b\) sobre \(R\), definimos
  \[ (a + b)_k = a_k + b_k \]
  y
  \[ (a \cdot b)_k = \sum_{i=0}^ka_{k-i}b_{i}\]
\end{definition}

Bajo está definición, no es difícil verificar que 
\( (R[[x]],+,\cdot) \) es un anillo conmutativo. 
Además, podemos distinguir a un elemento particular
de este conjunto, denominado
\emph{la indeterminada}, como
\[ x = (0,1,0,\ldots).\]
Para ésta, definimos también \( x^0 = 1\), \( x^1 = x\) y
de manera recusriva \( x^{m+1} = x \cdot x^m \).

\begin{proposition}
  Para cualquier pareja de enteros no negativos
  \( j\) y \( k\) se cumple:
  \[ (x^j)_k = \delta_{jk}.\]
\end{proposition}
\begin{proof}
  Procedemos por inducción sobre \( j\). Por definición
  \( x^0 = 1\) por lo que el resultado sigue. Supongamos
  \( (x^j)_k = \delta_{jk}\) para cualquier \( k\), entonces
  \begin{align*}
    (x^{j+1})_k &= (x \cdot x^{j})_k \\
    &= \sum_{i=0}^{k} x_{k-i}(x^{j})_{i} \\
    &= \sum_{i=0}^{k} x_{k-i}\delta_{ji} \\
    &= x_{k-j} \\
  \end{align*}
  Ademas,  \( x_{k-j} = \delta_{1(k-j)} = \delta_{(j+1)k}\),
  de lo que \( (x^{j+1})_k = \delta_{(j+1)k}\). Lo anterior
  es lo que buscábamos y por inducción la proposición sigue. Q.E.D.
\end{proof}

La proposición anterior, se puede usar para garantizar que
todos los elementos de \( R\left[ [x] \right]\)
se pueden expresar como ``combinaciones lineales''
del conjunto 
\[  \{1,x,x^2,\dots,x^n,\dots\}.\] 
Este hecho nos permite escribir una sucesión cualquiera
sobre \( R\),
\[ a = (a_0, a_1, \ldots)\]  como
\[ a = \sum_{i=0}^{\infty}a_ix^i.\]
Esto es un simple formalismo, nada se habla acerca 
de la convergencia de la suma y solamente hemos encontrado una
forma de expresar una sucesión en términos peculiares. 
Por la que se denomina a \( R\left[ [x] \right]\) como 
\emph{el anillo las series formales de potencia}.

\begin{proposition}
  \label{PR}
  Sea \( a = \sum_{i=0}^{\infty}a_ix^i\) una serie 
  formal de potencias en \( R\). Entonces, \( a\) es una 
  unidad en \( R\left[[x]\right]\) si y sólo si \( a_0\) es 
  una unidad en \( R\).
\end{proposition}
\begin{proof}
  Supongamos que \( a\) es una unidad, en ese caso debe existir 
  otra serie formal \( b = \sum_{i=0}b_ix^i \) de forma 
  que \( ab = 1\), o en otras palabras que 
  \( (a b)_k = \delta_{0k}\); en ese caso \( a_0 b_0 = 1\) por 
  lo que debemos concluir que \( a_0\) es una unidad.

  Si suponemos ahora que \( a_0\) es una unidad, entonces 
  existe \( u \in R\) tal que \( a_0 u= 1\),  definiremos la 
  serie formal  \( b\) de manera recursiva  
  \[ b_0 = u\]
  \[ b_{k} = -u \sum_{i=1}^{k}a_ib_{k-i} \]
  y afirmamos que ésta satisface \( a b = 1 \). En efecto
  \[ (ab)_0 = a_0b_0 = a_0u = 1\] 
  y para valores \( k\geq 1\), 
  de acuerdo a la definición que se provee 
  \[  -a_0b_k = \sum_{i=1}^{k}a_ib_{k-i} \]
  por lo que tenemos 
  \begin{align*}
    (ab)_k &= \sum_{i=0}^{k}a_{i}b_{k-i} \\
    	& = a_0b_k+ \sum_{i=1}^{k}a_{i}b_{k-i} \\
	& = 0
  \end{align*}
  como afirmamos. En ese caso, la serie formal de potencia 
  \( a\) presenta como inversa a la serie formal \( b\) por 
  lo que resulta una unidad en \( R\left[[x]\right]\)
  como buscábamos. Q.E.D.
\end{proof} 

\begin{example}
  Es común afirmar en una
  discusión analítica que
  \[ \frac{1}{1-x} = 1 + x + x^2 + \dots\]
  siendo la sucesión de la derecha una aproximación
  polinomial a la función de la izquierda. Para dar
  sentido a esta expresión se requiere dar un elaborado
  argumento de convergencia. Sin embargo, 
  a la sombra del resultado anterior, la sucesión
  \( (1-x)\) cumple con las hipótesis por lo que debe
  poseer una inversa en \( R\left[ [x] \right]\),
  la cual puede ser calculada identificando
  \[ 1 - x = (1,-1,0,0,\dots)\]
  lo cual resulta en obtener
  \[ (1-x)^{-1} = (1, 1, 1, \dots),\]
  lo cual se acostumbra representar como
  \[ \frac{1}{1-x} = 1 + x + x^2 + \dots.\]
  Es sorprendente que la definición \ref{OP} haga 
  posible hablar de manera simbólica de objetos sin
  mencionar siquiera la convergencia.
\end{example}

\end{document}


