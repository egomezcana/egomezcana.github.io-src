
\documentclass[draft,letter,10pt,notitlepage]{article}
\usepackage[utf8]{inputenc}
\usepackage{amsthm}
\usepackage{amsmath}
\usepackage{amsfonts}
\usepackage{amssymb}
%\usepackage{amscd}
\usepackage[spanish]{babel}
\usepackage{geometry}

%Tipografía distinta
\usepackage{amsopn}
\usepackage{mathrsfs}
\usepackage{palatino}
\usepackage{mathpazo}

\newtheorem{theorem}{Teorema}
\newtheorem{lemma}{Lema}
\newtheorem{proposition}{Proposición}
\newtheorem{corollary}{Corolario}

\theoremstyle{definition}
\newtheorem{definition}{Definición}
\newtheorem{axiom}{Axioma}
\newtheorem*{example}{Ejemplo}

\theoremstyle{remark}
\newtheorem*{remark}{Comentario}


\title{En defensa de la suma}
\date{3-May-2016}

\begin{document}
\maketitle
\section{Sumat\dots ¿qué?}
Un tema controvertido entre los matemáticos hispanohablantes, 
es como hemos de referirnos a la suma que involucra varios términos.
En general, la expresión \( a + b\) se admite como \emph{la suma de dos números}
y de manera similar  \( a + b + c\) como \emph{la suma de tres números}. 
No sólo eso, si escribimos
\[ a_1 + a_2 + \dots + a_n\]
es muy probable que se encuentre poca nula resistencia en
denominar a la expresión anterior como \emph{la suma de \( n\) números}.
Sin embargo, las cosas se tornan irremediablemente extrañas cuando 
queremos darle un significado preciso a la expresión anterior escribiendo
\[ \sum_{i=1}^n a_i.\]
A esta expresión, de manera paradójica, se le conoce como
\emph{la sumatoria de \( n\) números}. La pregunta es obligada, ¿por qué, si
tienen el mismo significado, se opta por referirnos a éstas expresiones de 
manera distinta?

\section{Orígenes}

La expresión «suma de dos números» significa la acción y efecto de sumar.

\section{Actualidad}
Actualmente, parece extenderse sin crítica alguna el uso y desuso del
término «sumatoria». Además, permea una especie de fatalismo justificado
en el indeleble y extendido uso del término. El problema está,
sin embargo, lejos de resolverse. El término por un lado, no está
aceptado por Real Academia de la Lengua Española y en contraste,
está reconocido el término «sumatorio».

\section{El argumento pedagógico}
Por último, quiero explorar un argumento difícil de discutir que tuve
la oportunidad de escuchar hace no mucho. Algunos proponentes del 
término, plantean que la operación de sumar muchos números es un
proceso abstracto mucho más elaborado que la suma de una cantidad
concreta de números y distinguirlo con una palabra
especial, consigue enfatizar su importancia y lo diferente que puede
resultar de la suma por todos conocida.

Hay poco que argumentar en contra de eso, coincido totalmente con
esa postura, el proceso de sumar dos números o tres o cuatro, es
de alguna forma mucho más simple comparado al proceso de sumar una 
secuencia arbitraria de números y el paso de uno a otro es toda una odisea
educativa. Pero bajo ninguna circunstancia este argumento favorece
al uso de un término ajeno al lenguaje. Al contrario, parece
sugerir que la solución debe encontrarse en el uso creativo
de la lengua.

Esto último, abre la puerta a una propuesta tan
natural que me parecía sorprendente que nadie hiciera antes. Por
un lado la palabra «sumatoria» no aparece en un diccionario así que,
en lugar de blindarse agregándola, hagamos uso de nuestro magnífico
idioma. Personalmente, no encuentro razón que nos pueda disuadir de 
expresarnos de la siguiente manera:
\begin{quote}
  Una suma de \( n\) números se dirá expresada  \emph{en notación
  \( \Sigma\) para la suma} si está escrita como
  \[ \sum_{i=1}^n a_i.\]
\end{quote}
De esta forma, hacemos un uso creativo del idioma para distinguir
un concepto abstracto manteniendo con esto el argumento pedagógico.
Una expresión de este estilo es mucho más deseable desde el punto
de vista lingüístico al no requerir reforma alguna a la lengua.
Además, se puede extender de manera muy natural, por ejemplo,
para el producto:
\begin{quote}
  Un producto de \( n\) números se dirá expresado  \emph{en notación
  \( \Pi\) para el producto} si está escrito como
  \[ \prod_{i=1}^n a_i.\]
\end{quote}
Lo anterior es preferible al término «productoria» que parece comenzar
a tener un preocupante apego y que a diferencia de la suma, no tiene
correspondiente anglosajón. Esto quiere decir, que no existe una distinción
lingüística entre la acción y efecto de multiplicar números.

\section{Conclusiones}
El uso del término «sumatoria» no es irremediable,
es posible proponer alternativas sencillas y creativas que pueden ser 
aceptadas en la escena técnica y lingüística por igual. Aunque parezca 
una discusión estéril, como mujeres y hombres de ciencia no podemos 
renunciar al uso adecuado de las herramientas que nos rodean y una 
imprescindible es por supuesto la lengua. El caso de la suma es un ejemplo
de un fenómeno que se ha ido extendiendo en los últimos años y que invita a
cuestionarnos cuan deseable es el uso de anglicismos en lugar de palabras
equivalentes en significado procedentes del español. En mi opinión, 
su uso parece justificado en un pragmatismo endeble y éste debería ser
rebasado por una simple, pero real, necesidad: Hablar en español.

\begin{thebibliography}{}

  \bibitem{Erdos01} asdas 

\end{thebibliography}
\end{document}


