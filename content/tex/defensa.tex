
\documentclass[draft,letter,10pt,notitlepage]{article}
\usepackage[utf8]{inputenc}
\usepackage{amsthm}
\usepackage{amsmath}
\usepackage{amsfonts}
\usepackage{amssymb}
%\usepackage{amscd}
\usepackage[spanish]{babel}
\usepackage{geometry}

%Tipografía distinta
\usepackage{amsopn}
\usepackage{mathrsfs}
\usepackage{palatino}
\usepackage{mathpazo}

\newtheorem{theorem}{Teorema}
\newtheorem{lemma}{Lema}
\newtheorem{proposition}{Proposición}
\newtheorem{corollary}{Corolario}

\theoremstyle{definition}
\newtheorem{definition}{Definición}
\newtheorem{axiom}{Axioma}
\newtheorem*{example}{Ejemplo}

\theoremstyle{remark}
\newtheorem*{remark}{Comentario}


\title{En defensa de la suma}
\date{3-May-2016}

\begin{document}
\maketitle
\section{Sumat\dots ¿qué?}
Un tema controvertido entre los matemáticos hispanohablantes, 
resulta el cómo hemos de referirnos a la suma que involucra varios términos.
En general, la expresión \( a + b\) se admite como \emph{la suma de dos números}
y de manera similar  \( a + b + c\) como \emph{la suma de tres números}. 
No sólo eso, si escribimos
\[ a_1 + a_2 + \dots + a_n\]
es muy probable que se encuentre poca o nula resistencia en
denominar a la expresión anterior como \emph{la suma de \( n\) números}.
Sin embargo, las cosas se tornan irremediablemente extrañas cuando 
queremos darle un significado preciso a la expresión anterior escribiendo
\[ \sum_{i=1}^n a_i.\]
A esta expresión, de manera paradójica, se le conoce como
\emph{la sumatoria de \( n\) números}. La pregunta es obligada, ¿por qué, si
tienen el mismo significado, se opta por referirnos a éstas expresiones de 
manera distinta?

\section{Orígenes}

La palabra «suma», según la RAE significa «acción y efecto de
sumar». Esto quiere decir que «suma» tiene una acepción como verbo y
otra como sustantivo. Sin embargo, en las lenguas sajonas, como el
inglés, existen palabras distintas para describir la acción y el
efecto de sumar. Así, es común encontrar el término «summation» cuando
se indica el proceso de una suma y «sum» cuando se obtiene el valor de
una. Según el significado la palabra suma, la distinción inglesa de
los términos es completamente redundante en español. A pesar de esto,
algunos traductores durante los años ochenta, creyeron que el
significado de la palabra «summation» no era parte del español y
decidieron castellanizar dicha palabra como «sumatoria». Desafortunado
o no, estás traducciones fueron usadas largamente por tratarse de
materiales de primerísima calidad (como los dos volúmenes de
\textit{Calculus} de Tom Apostol) y los cuales son ampliamente
usados en las facultades del mundo hispano. Esto, por supuesto, ayudo
a extender el uso de la palabra hasta su estado actual.

\section{Actualidad}
Actualmente, parece extenderse sin crítica alguna el uso y desuso del
término «sumatoria». Además, permea una especie de fatalismo justificado
en el indeleble y extendido uso del término. El problema está,
sin embargo, lejos de resolverse. El término por un lado, no está
aceptado por Real Academia de la Lengua Española y en contraste,
está reconocido el término «sumatorio» (lo cual, sólo consigue crear
más confusión) y por el otro, su origen poco claro indudablemente le
resta legitimidad. Hay varios argumentos para justificar su uso y aquí
abordaremos dos de los más socorridos.

\section{El argumento de uso}
El principal argumento a favor del uso del término está basado en
una forma fatalismo que nos pide aceptar los cambios en el español. Su
argumento, que podemos llamar argumento de uso, razona de la siguiente
manera:

\begin{itemize}
\item[Pr. 1.] Si una palabra no aceptada alcanza un uso extendido,
  entonces debe aceptarse como propia del idioma.
\item[Pr. 2.] La palabra «sumatoria» tiene un uso extendido.
\item[Con.] La palabra «sumatoria» debe aceptarse como propia del idioma.
\end{itemize}

Aunque el argumento anterior es válido, parece probar demasiado pues
no es difícil imaginarse otra palabra que debería ser aceptada sólo
por tener un uso extendido. Esto por supuesto, depende del significado
que pretendamos dar al uso extendido de una palabra. De hecho este es
el caso, en México tenemos una tendencia de modificar

Los argumentos de esa forma fallan pues los cambios en el idioma no
suceden de manera tan sencilla, al menos hay un componente que el
argumento parece ignorar: La necesidad. Algo que comparten la palabra
«sumatoria» y el verbo «washear» es la nula necesidad de su existencia.
En el fondo, esto parece mostrar que considerar al uso como una
condición suficiente para generar cambios en un idioma es algo
exagerado y en realidad debería ser (por lo menos) una combinación de
uso y necesidad. Habitualmente, nos topamos con objetos, verbos o
situaciones que no tienen una forma precisa de expresarse en
español. En esos casos, el uso de un vocablo fuera del español está
completamente justificado. Por ejemplo, posterior a la conquista de
México, en Europa no se conocía la fruta tropical que los indígenas
denominaban «aguacatl». El término fue hispanizado como «aguacate»,
obedeciendo a la necesidad que existía de nombrar aquella fruta,
desconocida en ese momento, en español. No fue capricho de nadie, ni
un asunto sólo de uso. Cuando existe la necesidad de nombrar un
objeto, verbo o circunstancia del que no tenemos precedente y del
que no podemos dar una denominación en español, podemos proponer un
término hispanizado. Una vez que éste término se ha extendido lo
suficiente, podemos pensar en agregarlo al idioma. Esto, sin embargo,
no es el caso con el término «sumatoria». Por muy extendido que
parezca su uso, no hay una real necesidad de éste a razón que su
origen parece surgir de una extraña traducción de algunos textos.
Al menos, no parecería razonable permitir que éste justificado su uso
y ese es el principal problema.

\section{El argumento pedagógico}
A últimas fechas, encontré un argumento adicional al de uso y el cual
encuentro difícil de discutir. Algunos proponentes del término,
plantean que la operación de sumar muchos números es un
proceso abstracto mucho más elaborado que la suma de una cantidad
concreta de números y distinguirlo con una palabra
especial, consigue enfatizar su importancia y lo diferente que puede
resultar de la suma por todos conocida.

Hay poco que argumentar en contra de eso y personalmente comulgo con
dicha postura, el proceso de sumar dos números o tres o cuatro, es
de alguna forma mucho más simple comparado al proceso de sumar una 
secuencia arbitraria de números y el paso de uno a otro es toda una odisea
intelectual (aunque no considero que sea extremadamente complicado).
Bajo ninguna circunstancia, sin embargo, este argumento favorece
al uso de un término ajeno al lenguaje. Al contrario, parece
sugerir que la solución debe encontrarse en el uso creativo
de la lengua y no en su menosprecio. Esto último, abre la puerta a una
propuesta tan natural y obvia que parece una necedad exponerla. Por
un lado la palabra «sumatoria» no aparece en un diccionario así que,
en lugar de blindarse agregándola, hagamos uso de nuestro magnífico
idioma. Personalmente, no encuentro razón que nos pueda disuadir de 
expresarnos de la siguiente manera:
\begin{quote}
  Una suma de \( n\) números se dirá que está expresada \emph{en notación
  \( \Sigma\) para la suma}, si está escrita como  \[\sum_{i=1}^n a_i.\]
\end{quote}
De esta forma, hacemos un uso creativo del idioma para distinguir
un concepto abstracto manteniendo con esto el argumento pedagógico.
Una expresión de este estilo es mucho más deseable desde el punto
de vista lingüístico al no requerir reforma alguna a la lengua.
Además, se puede extender de manera muy natural, por ejemplo,
para el producto:
\begin{quote}
  Un producto de \( n\) números se dirá expresado  \emph{en notación
  \( \Pi\) para el producto} si está escrito como
  \[ \prod_{i=1}^n a_i.\]
\end{quote}
Lo anterior es preferible al término «productoria» que parece comenzar
a tener un preocupante apego y que a diferencia de la suma, no tiene
correspondiente anglosajón. Esto quiere decir, que no existe una distinción
lingüística entre la acción y efecto de multiplicar números, ni
siquiera en el inglés por lo que sería absolutamente innecesario crear
una en español.

\section{Conclusiones}
El uso del término «sumatoria» no es irremediable, como algunas
posiciones fatalistas parecen empeñarse en creer. Es posible y más
aun, sencillo, proponer alternativas que puedan ser aceptadas en la
escena técnica y lingüística por igual. Aunque parezca una discusión
estéril, como mujeres y hombres de ciencia no podemos renunciar al uso
adecuado de las herramientas que nos rodean y una imprescindible es
por supuesto la lengua. El caso de la suma es un ejemplo de un
fenómeno que se ha ido extendiendo en los últimos años y que invita a
cuestionar cuan deseable es el uso de anglicismos en lugar de palabras
equivalentes en significado pero procedentes del español. En mi
opinión, su uso parece justificado en un pragmatismo demasiado endeble
y éste debería ser rebasado por una simple, pero real, necesidad:
Hablar en español.

\begin{thebibliography}{}

  \bibitem{Erdos01} asdas 

\end{thebibliography}
\end{document}



%%% Local Variables:
%%% mode: latex
%%% TeX-master: t
%%% End:
